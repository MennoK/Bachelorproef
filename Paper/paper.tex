\typeout{Bewegingsherkenning met een smartphone}

\documentclass{article}
\usepackage[dutch]{babel}
\usepackage{ijcai11}
\usepackage{times}
%\usepackage{latexsym}  %optional

\title{Bewegingsherkenning met een smartphone}
\author{Arne De Brabandere\\
	arne.debrabandere@student.kuleuven.be
    \And
    Menno Keustermans\\
    menno.keustermans@student.kuleuven.be}

\begin{document}

\maketitle

\begin{abstract}

%TODO doel + probleemstelling + belangrijkste resultaten

\end{abstract}

\section{Inleiding}

%TODO situering van het werk + bijdragen:
% - waarom bewegingen herkennen? (extra bronnen gebruiken
% - waarom met smartphones?
% - korte uitleg over onderzoek dat al gedaan is hierover voor afzonderlijke activiteiten (zie papers van literatuurstudie)
% - uitleg over eigen bijdrage: (1) van model om afzonderlijke activiteiten te herkennen naar algoritme om sequenties van activiteiten te herkennen (+ duidelijk zeggen wat we met afzondelijke activiteiten en sequenties bedoelen)



% bij 2 secties telkens "data collectie" + "data verwerking" + leermethodes + experimenten  +resultaten + conclusie

\section{Afzondelijke activiteiten}

Het eerste probleem is om van een gegeven reeks samples van de accelerometer en gyroscoop van een smartphone de activiteit van een persoon te bepalen. We veronderstellen hier dat telkens \'e\'en afzonderlijke activiteit gemeten wordt.

We willen 10 verschillende activiteiten kunnen herkennen:
\begin{itemize}
\item wandelen,
\item lopen,
\item fietsen,
\item een trap opwandelen,
\item een trap afwandelen,
\item springen,
\item niets doen (zitten, liggen, staan),
\item een lift naar boven nemen, %TODO anders verwoorden? (want eigenlijk is het alleen de versnelling van een lift naar boven, en lift naar boven = versnelling naarboven + versnelling naar beneden)
\item een lift naar beneden nemen,
\item tanden poetsen.
\end{itemize}

In bovenstaande lijst hebben we tanden poetsen als moeilijke activiteit toegevoegd. De beweging lijkt sterk op niets doen en zal waarschijnlijk minder goed te herkennen zijn. Hetzelfde geldt voor de lift naar boven/beneden nemen.

Met behulp van classificatiemethodes zoeken we naar een model met een zo groot mogelijke accuraatheid om de activiteit van een meting te bepalen.

%TODO ook motivatie (waarom classificatiemethodes)

\subsection{Datacollectie}

We hebben voor elke activiteit 22 metingen verzameld, opgemeten door 2 verschillende personen. Om voldoende variatie te hebben, gebeurden de metingen op verschillende dagen. Ook hebben we ervoor gezorgd dat we niet telkens dezelfde broek droegen, aangezien de gemeten versnelling kan vari\"eren in verschillende broekzakken. Elke meting bevat bovendien exact \'e\'en activiteit.

De metingen werden met de MotionTracker
%TODO verwijzing met voetnoot: geschreven door ...
tool gedaan. Dit is een Android-applicatie die de versnelling en rotatie (gemeten door de accelerometer en gyroscoop van de smartphone) aan 50 Hz samplet. Als uitvoer geeft de applicatie een logbestand met de gemeten versnellingen (in de x-as, y-as en z-as met de z-as evenwijdig met de gravitatie) en rotaties (in quaternion notatie) met bijhorende timestamps.

\subsection{Dataverwerking: features berekenen}

Voor we classificatiemethodes kunnen gebruiken, moeten we eerst features berekenen. Dit zijn parameters die we uit de samples van de accelerometer en gyroscoop kunnen halen.
%TODO uitleggen waarom we features berekenen

We gebruiken 4 soorten features:
\begin{itemize}
\item Statistische features: gemiddelde, standaardafwijking van versnelling en vermogen
\item Fourier-transformatie: amplitudes horende bij bepaalde frequenties
\item Wavelet-transformatie: ... %TODO
\item Hidden Markov models: log-likelihood voor het model van elke activiteit
\end{itemize}

\subsection{Classificatiemethodes}

We gebruiken classificatiemethodes om een model te zoeken om afzonderlijke activiteiten te herkennen. We vergelijken enkele veel voorkomende methodes: beslissingsbomen, Random Forest, k-Nearest Neighbours, Naive Bayes en Support Vector Machines. De methodes leren telkens een model uit de instanties met de hierboven beschreven features voor de verschillende activiteiten. Hieronder wordt kort de werking van de vernoemde methodes beschreven:

\begin{itemize}
\item \textit{Beslissingsbomen} zijn bomen waarvan de interne knopen features voorstellen en de bladknopen \'e\'en (of meerdere) labels bevatten. Een tak in de boom stelt een test voor op de feature van de knoop waaruit de tak vertrekt. Om een instantie te classificeren worden de juiste takken gevolgd tot in een bepaalde bladknoop.
\end{itemize}

%TODO methodes uitleggen

\subsection{Experimenten en resultaten}

We evalueren de methodes met 10-fold cross-validatie. Hierbij worden de instanties op 10 verschillende manieren opgesplitst zodat telkens 90\% van de instanties als trainingset wordt gebruikt om een model uit te leren. Op de overige 10\% wordt het model ge\"evalueerd. De accuraatheid van elke methode wordt dan berekend als het gemiddelde percentage van correct geclassificeerde instanties in de verschillende testsets.

In figuur X %TODO staafdiagram invoegen met vergelijking van methodes
wordt de accuraatheid van de verschillende methodes vergeleken. We zien dat Random Forest de hoogste accuraatheid geeft voor onze metingen. Ook beslissingsbomen (J48) geeft een goede accuraatheid, net zoals Naive Bayes. De hogere accuraatheid van Random Forest ten opzichte van beslissingsbomen is waarschijnlijk te verklaren door het feit dat Random Forest minder kans op overfitting heeft. %TODO hier verwijzen naar paper (Bronnen/randomforest2001.pdf)


%TODO experimenten + resultaten: opsplitsen?

\section{Sequenties van activiteiten}

%TODO probleemstelling

\subsection{Datacollectie}

%TODO hoe hebben we de data verzameld

\subsection{Dataverwerking}

%TODO iets over het labelen + uitleg van tijdsvensters, overlappingen

\subsection{Algoritme}

%TODO algoritme uitleggen + hierbij ook uitleg over ruis-cutoff

\subsection{Experimenten en resultaten}

%TODO
% - hoe gebeurt de evaluatie?
% - accuraatheid plotten in functie van grootte van tijdsvensters en overlap + tijdsvensters vergelijken + iets over ruis cut-off
%   ==> hypothese (waarschijnlijk 4sec en 3/4 overlap, omwille van activiteit lift versnelt omhoog/omlaag)
% - bespreking van resultaten: komt dit overeen met de hypothese?

\section{Conclusie}

%TODO voor afzonderlijke activiteiten EN (?) voor sequenties

\section{Verder werk}

%TODO misschien wat er zou gebeuren als hetzelfde zou gedaan worden voor meer metingen? (bvb. wandelen, lopen, ... accurater)




%TODO bronvermelding!

%% The file named.bst is a bibliography style file for BibTeX 0.99c
% \bibliographystyle{named}
% \bibliography{paper}

\end{document}

