\typeout{Bewegingsherkenning met een smartphone}

\documentclass{article}
\usepackage[dutch]{babel}
\usepackage{ijcai11}
\usepackage{times}
%\usepackage{latexsym}  %optional

\title{Bewegingsherkenning met een smartphone}
\author{Arne De Brabandere\\
	arne.debrabandere@student.kuleuven.be
    \And
    Menno Keustermans\\
    menno.keustermans@student.kuleuven.be}

\begin{document}

\maketitle

\begin{abstract}

% TODO: doel + probleemstelling + belangrijkste resultaten

\end{abstract}

\section{Inleiding}

% TODO: situering van het werk + bijdragen:
% - waarom bewegingen herkennen? (extra bronnen gebruiken
% - waarom met smartphones?
% - korte uitleg over onderzoek dat al gedaan is hierover voor afzonderlijke activiteiten (zie papers van literatuurstudie)
% - uitleg over eigen bijdrage: (1) van model om afzonderlijke activiteiten te herkennen naar algoritme om sequenties van activiteiten te herkennen (+ duidelijk zeggen wat we met afzondelijke activiteiten en sequenties bedoelen)



% bij 2 secties telkens "data collectie" + "data verwerking" + leermethodes + experimenten  +resultaten + conclusie

\section{Afzondelijke activiteiten}

Het eerste probleem is om van een gegeven reeks samples van een accelerometer en een gyroscoop de activiteit te bepalen. We eisen hier dat telkens \'e\'en afzonderlijke activiteit gemeten wordt. We willen 10 verschillende activiteiten kunnen herkennen:
\begin{itemize}
\item wandelen,
\item lopen,
\item fietsen,
\item een trap opwandelen,
\item een trap afwandelen,
\item een lift naar boven nemen,
\item een lift naar beneden nemen,
\item tanden poetsen,
\item springen,
\item niets doen (zitten, liggen, staan).
\end{itemize}

Met behulp van classificatiemethodes zoeken we naar een model met een zo groot mogelijke accuraatheid om de activiteit van een meting te bepalen.

% TODO: ook motivatie (waarom classificatiemethodes)

\subsection{Datacollectie}

% probleemstelling:
Hiervoor hebben we voor elke activiteit 22 metingen verzameld, opgemeten door 2 verschillende personen. De metingen gebeurden in vari\"erende omstandigheden zoals tijd en kledij. Voor elke meting werd gezorgd dat die slechts \'e\'en activiteit bevat.
% ...

\subsection{Dataverwerking: features berekenen}

Vooraleer we classificatiemethodes kunnen gebruiken, moeten we eerst features berekenen. Dit zijn parameters die we uit de samples van de accelerometer en gyroscoop kunnen halen. 

% TODO: uitleggen waarom we features berekenen

\subsection{Classificatiemethodes}

% TODO: korte uitleg van belangrijke (goed werkende) methodes (extra bronnen gebruiken)

\subsection{Experimenten en resultaten}

% TODO: experimenten + resultaten: opsplitsen?

\section{Sequenties van activiteiten}

% TODO: probleemstelling

\subsection{Datacollectie}

% TODO: hoe hebben we de data verzameld

\subsection{Dataverwerking}

% TODO: iets over het labelen + uitleg van tijdsvensters, overlappingen

\subsection{Algoritme}

% TODO: algoritme uitleggen + hierbij ook uitleg over ruis-cutoff

\subsection{Experimenten en resultaten}

% TODO:
% - hoe gebeurt de evaluatie?
% - accuraatheid plotten in functie van grootte van tijdsvensters en overlap + tijdsvensters vergelijken + iets over ruis cut-off
%   ==> hypothese (waarschijnlijk 4sec en 3/4 overlap, omwille van activiteit lift versnelt omhoog/omlaag)
% - bespreking van resultaten: komt dit overeen met de hypothese?

\section{Conclusie}

% TODO: voor afzonderlijke activiteiten EN (?) voor sequenties

\section{Verder werk}

% TODO: misschien wat er zou gebeuren als hetzelfde zou gedaan worden voor meer metingen? (bvb. wandelen, lopen, ... accurater)




% TODO: bronvermelding!

%% The file named.bst is a bibliography style file for BibTeX 0.99c
% \bibliographystyle{named}
% \bibliography{paper}

\end{document}

