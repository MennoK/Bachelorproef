\typeout{Bewegingsherkenning met een smartphone}

\documentclass{article}
\usepackage[dutch]{babel}
\usepackage{ijcai11}
\usepackage{times}
%\usepackage{latexsym}  %optional

\title{Bewegingsherkenning met een smartphone}
\author{Arne De Brabandere\\
	arne.debrabandere@student.kuleuven.be
    \And
    Menno Keustermans\\
    menno.keustermans@student.kuleuven.be}

\begin{document}

\maketitle

\begin{abstract}

\textit{
\begin{itemize}
\item doel
\item probleemstelling
\item belangrijkste resultaten
\end{itemize}
}

\end{abstract}

\section{Inleiding}

\textit{
Situering van het werk + bijdragen:
\begin{itemize}
\item waarom bewegingen herkennen? (extra bronnen gebruiken)
\item waarom met smartphones?
\item korte uitleg over onderzoek dat al gedaan is hierover voor afzonderlijke activiteiten (zie papers van literatuurstudie)
\item uitleg over eigen bijdrage: van model om afzonderlijke activiteiten te herkennen naar algoritme om sequenties van activiteiten te herkennen
\end{itemize}
}

\section{Afzonderlijke activiteiten}

\textit{
\begin{itemize}
\item probleemstelling
\item hoe? (classificatie methodes)
\item waarom eerst dit voor sequenties? (al iets zeggen over tijdsvensters en dergelijke)
\end{itemize}
}

\subsection{Classificatie methodes}

\textit{
\begin{itemize}
\item uitleg van belangrijke (goed werkende) methodes (extra bronnen gebruiken),
telkens samen met resultaten + bespreking waarom goed of niet
\item besluit voor afzonderlijke activiteiten
\end{itemize}
}

\section{Sequenties van activiteiten}

\textit{probleemstelling}

\subsection{Oplossing}

\textit{hoe? (terug over tijdsvensters, meer uitgebreid: verschillende groottes en overlappingen + hypothese (waarschijnlijk 4sec en 3/4 overlap, omwille van activiteit \textbf{lift versnelt omhoog/omlaag}) ...)}

\subsection{Evaluatie}

\textit{
\begin{itemize}
\item hoe gebeurt de evaluatie: \textbf{nog eens bekijken! (overgangen niet meerekenen)}
\item accuraatheid plotten in functie van grootte van tijdsvensters, overlap en gebruikte model
\item bespreking van resultaten: komt dit overeen met de hypothese?
\end{itemize}
}

\section{Verder werk}

\textit{misschien wat er zou gebeuren als hetzelfde zou gedaan worden voor meer metingen?}

\section{Conclusie}


% wat er nog niet in staat: hoe we de metingen gedaan hebben (voor afzonderlijke activiteiten vooral, en ook voor sequenties)



%% The file named.bst is a bibliography style file for BibTeX 0.99c
% \bibliographystyle{named}
% \bibliography{paper}

\end{document}

