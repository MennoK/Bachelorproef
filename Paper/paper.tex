\typeout{Bewegingsherkenning met een smartphone}

\documentclass{article}
\usepackage[dutch]{babel}
\usepackage{ijcai11}
\usepackage{times}
%\usepackage{latexsym}  %optional

\title{Bewegingsherkenning met een smartphone}
\author{Arne De Brabandere\\
	arne.debrabandere@student.kuleuven.be
    \And
    Menno Keustermans\\
    menno.keustermans@student.kuleuven.be}

\begin{document}

\maketitle

\begin{abstract}

\textit{
\begin{itemize}
\item doel
\item probleemstelling
\item belangrijkste resultaten
\end{itemize}
}

\end{abstract}

\section{Inleiding}

\textit{
Situering van het werk + bijdragen:
\begin{itemize}
\item waarom bewegingen herkennen? (extra bronnen gebruiken)
\item waarom met smartphones?
\item korte uitleg over onderzoek dat al gedaan is hierover voor afzonderlijke activiteiten (zie papers van literatuurstudie)
\item uitleg over eigen bijdrage: (1) van model om afzonderlijke activiteiten te herkennen naar algoritme om sequenties van activiteiten te herkennen (+ duidelijk zeggen wat we met afzondelijke activiteiten en sequenties bedoelen)
\end{itemize}
}

\section{Herkenning van afzondelijke activiteiten}

Het eerste probleem is om van een gegeven reeks samples van een accelerometer en een gyroscoop de activiteit te bepalen. We eisen hier dat telkens \'e\'en afzonderlijke activiteit gemeten wordt. We willen 10 verschillende activiteiten kunnen herkennen:
\begin{itemize}
\item wandelen,
\item lopen,
\item fietsen,
\item een trap opwandelen,
\item een trap afwandelen,
\item een lift naar boven nemen,
\item een lift naar beneden nemen,
\item tanden poetsen,
\item springen,
\item niets doen (zitten, liggen, staan).
\end{itemize}

Met behulp van classificatiemethodes zoeken we naar een model met een zo groot mogelijke accuraatheid om de activiteit van een meting te bepalen. Hiervoor hebben we voor elke activiteit 22 metingen verzameld, opgemeten door 2 verschillende personen. De metingen gebeurden in vari\"erende omstandigheden zoals tijd en kledij. Voor elke meting werd gezorgd dat die slechts \'e\'en activiteit bevat.

\textit{
\begin{itemize}
\item probleemstelling
\item hoe? (classificatie methodes) + motivatie
\end{itemize}
}

\subsection{Features}

Vooraleer we classificatiemethodes kunnen gebruiken, moeten we eerst features berekenen. Dit zijn parameters die we uit de samples van de accelerometer en gyroscoop kunnen halen. \textbf{[TODO: uitleggen waarom we features berekenen]}

\subsection{Classificatiemethodes}

\textit{
\begin{itemize}
\item uitleg van belangrijke (goed werkende) methodes (extra bronnen gebruiken),
telkens samen met resultaten + bespreking waarom goed of niet
\item besluit voor afzonderlijke activiteiten
\end{itemize}
}

\section{Sequenties van activiteiten}

\textit{probleemstelling}

\subsection{Oplossing}

\textit{hoe? (terug over tijdsvensters, meer uitgebreid: verschillende groottes en overlappingen + hypothese (waarschijnlijk 4sec en 3/4 overlap, omwille van activiteit \textbf{lift versnelt omhoog/omlaag}) ...)}

\subsection{Evaluatie}

\textit{
\begin{itemize}
\item hoe gebeurt de evaluatie: \textbf{nog eens bekijken! (overgangen niet meerekenen)}
\item accuraatheid plotten in functie van grootte van tijdsvensters, overlap en gebruikte model
\item bespreking van resultaten: komt dit overeen met de hypothese?
\end{itemize}
}

\section{Verder werk}

\textit{misschien wat er zou gebeuren als hetzelfde zou gedaan worden voor meer metingen?}

\section{Conclusie}


% wat er nog niet in staat: hoe we de metingen gedaan hebben (voor afzonderlijke activiteiten vooral, en ook voor sequenties)



%% The file named.bst is a bibliography style file for BibTeX 0.99c
% \bibliographystyle{named}
% \bibliography{paper}

\end{document}

